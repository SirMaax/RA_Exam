% Hinweise:
% - Dateinamen anpassen!
% - Name, Vorname, Studiengang, Matrikelnummer anpassen!
% - Hinweis zu Umlauten beachten (s.u.)!
% - Spellchecker verwenden!
% - PDF erzeugen: pdflatex ausarbeitung_emustermann.tex

\documentclass[12pt]{article}
\usepackage[a4paper]{anysize}
\usepackage[onehalfspacing]{setspace}
% LRTB
%\marginsize{3.0cm}{3.0cm}{2.0cm}{2.0cm}
\usepackage[german,english]{babel}
\usepackage{times}
\usepackage[utf8]{inputenc}
\usepackage{layouts}
\usepackage{url}
\begin{document}
\selectlanguage{german}
\newcommand{\3}{\ss}

\begin{center}
  \noindent{\large\bf%
    Ausarbeitung Rechnerarchitektur Wintersemester 2021/22}\\[1cm]
  \noindent{\Large\bf%
    Assembler-Implementierung eines bitweisen LSB-Radix-Sort}\\[3cm]
\end{center}

\noindent{\bf Name:} Mustermann\\

\noindent{\bf Vorname:} Erika\\

\noindent{\bf Studiengang:} Angewandte Scholastik\\

\noindent{\bf Matrikelnummer:} 0123456789\\[3cm]

\noindent Hiermit erkläre ich, dass ich das Programm und die
vorliegende Ausarbeitung selbstständig verfasst habe. Ich habe keine
anderen Quellen als die angegebenen benutzt und habe die Stellen in
Programm und in der Ausarbeitung, die anderen Quellen entnommen
wurden, in jedem Fall unter Angabe der Quelle als Entlehnung kenntlich
gemacht. Diese Erklärung gilt auch ohne meine Unterschrift, sobald ich
das Programm und die Ausarbeitung über die E-Prüfung der Vorlesung
Rechnerarchitektur im Lernraum des eKVV an der Universität Bielefeld
und unter Angabe meiner Matrikelnummer in der Ausarbeitung eingereicht
habe.

\thispagestyle{empty}
\newpage
\setcounter{page}{1}

``So macht man Anführungszeichen'', "aber so geht es nicht".

\section{Implementation des gesamten Sortiervorgangs}
%====================================================

Fügen Sie hier Ihren Text ein. Fügen Sie hier Ihren Text ein. Fügen
Sie hier Ihren Text ein. Fügen Sie hier Ihren Text ein. Fügen Sie hier
Ihren Text ein. Fügen Sie hier Ihren Text ein. Fügen Sie hier Ihren
Text ein. Fügen Sie hier Ihren Text ein. Fügen Sie hier Ihren Text
ein. Fügen Sie hier Ihren Text ein. Fügen Sie hier Ihren Text
ein. Fügen Sie hier Ihren Text ein. Fügen Sie hier Ihren Text
ein. Fügen Sie hier Ihren Text ein. Fügen Sie hier Ihren Text
ein. Fügen Sie hier Ihren Text ein. Fügen Sie hier Ihren Text
ein. Fügen Sie hier Ihren Text ein.

\section{Implementation des Sortierdurchgangs für ein Bit}
%=========================================================

Fügen Sie hier Ihren Text ein. Fügen Sie hier Ihren Text ein. Fügen
Sie hier Ihren Text ein. Fügen Sie hier Ihren Text ein. Fügen Sie hier
Ihren Text ein. Fügen Sie hier Ihren Text ein. Fügen Sie hier Ihren
Text ein. Fügen Sie hier Ihren Text ein. Fügen Sie hier Ihren Text
ein. Fügen Sie hier Ihren Text ein. Fügen Sie hier Ihren Text
ein. Fügen Sie hier Ihren Text ein. Fügen Sie hier Ihren Text
ein. Fügen Sie hier Ihren Text ein. Fügen Sie hier Ihren Text
ein. Fügen Sie hier Ihren Text ein. Fügen Sie hier Ihren Text
ein. Fügen Sie hier Ihren Text ein.

\section{Beibehalten der Reihenfolge bei gleichem Bit}
%======================================================

Fügen Sie hier Ihren Text ein. Fügen Sie hier Ihren Text ein. Fügen
Sie hier Ihren Text ein. Fügen Sie hier Ihren Text ein. Fügen Sie hier
Ihren Text ein. Fügen Sie hier Ihren Text ein. Fügen Sie hier Ihren
Text ein. Fügen Sie hier Ihren Text ein. Fügen Sie hier Ihren Text
ein. Fügen Sie hier Ihren Text ein. Fügen Sie hier Ihren Text
ein. Fügen Sie hier Ihren Text ein. Fügen Sie hier Ihren Text
ein. Fügen Sie hier Ihren Text ein. Fügen Sie hier Ihren Text
ein. Fügen Sie hier Ihren Text ein. Fügen Sie hier Ihren Text
ein. Fügen Sie hier Ihren Text ein.

\section{Behandlung von Vorzeichen und Betrag der Keys}
%======================================================

Fügen Sie hier Ihren Text ein. Fügen Sie hier Ihren Text ein. Fügen
Sie hier Ihren Text ein. Fügen Sie hier Ihren Text ein. Fügen Sie hier
Ihren Text ein. Fügen Sie hier Ihren Text ein. Fügen Sie hier Ihren
Text ein. Fügen Sie hier Ihren Text ein. Fügen Sie hier Ihren Text
ein. Fügen Sie hier Ihren Text ein. Fügen Sie hier Ihren Text
ein. Fügen Sie hier Ihren Text ein. Fügen Sie hier Ihren Text
ein. Fügen Sie hier Ihren Text ein. Fügen Sie hier Ihren Text
ein. Fügen Sie hier Ihren Text ein. Fügen Sie hier Ihren Text
ein. Fügen Sie hier Ihren Text ein.

\section{Prüfung einzelner Bits im Datenwort}
%============================================

Fügen Sie hier Ihren Text ein. Fügen Sie hier Ihren Text ein. Fügen
Sie hier Ihren Text ein. Fügen Sie hier Ihren Text ein. Fügen Sie hier
Ihren Text ein. Fügen Sie hier Ihren Text ein. Fügen Sie hier Ihren
Text ein. Fügen Sie hier Ihren Text ein. Fügen Sie hier Ihren Text
ein. Fügen Sie hier Ihren Text ein. Fügen Sie hier Ihren Text
ein. Fügen Sie hier Ihren Text ein. Fügen Sie hier Ihren Text
ein. Fügen Sie hier Ihren Text ein. Fügen Sie hier Ihren Text
ein. Fügen Sie hier Ihren Text ein. Fügen Sie hier Ihren Text
ein. Fügen Sie hier Ihren Text ein.

\section{Sortieren in beide Richtungen}
%======================================

Fügen Sie hier Ihren Text ein. Fügen Sie hier Ihren Text ein. Fügen
Sie hier Ihren Text ein. Fügen Sie hier Ihren Text ein. Fügen Sie hier
Ihren Text ein. Fügen Sie hier Ihren Text ein. Fügen Sie hier Ihren
Text ein. Fügen Sie hier Ihren Text ein. Fügen Sie hier Ihren Text
ein. Fügen Sie hier Ihren Text ein. Fügen Sie hier Ihren Text
ein. Fügen Sie hier Ihren Text ein. Fügen Sie hier Ihren Text
ein. Fügen Sie hier Ihren Text ein. Fügen Sie hier Ihren Text
ein. Fügen Sie hier Ihren Text ein. Fügen Sie hier Ihren Text
ein. Fügen Sie hier Ihren Text ein.

\section{Laufzeitanalyse}
%========================

Fügen Sie hier Ihren Text ein. Fügen Sie hier Ihren Text ein. Fügen
Sie hier Ihren Text ein. Fügen Sie hier Ihren Text ein. Fügen Sie hier
Ihren Text ein. Fügen Sie hier Ihren Text ein. Fügen Sie hier Ihren
Text ein. Fügen Sie hier Ihren Text ein. Fügen Sie hier Ihren Text
ein. Fügen Sie hier Ihren Text ein. Fügen Sie hier Ihren Text
ein. Fügen Sie hier Ihren Text ein. Fügen Sie hier Ihren Text
ein. Fügen Sie hier Ihren Text ein. Fügen Sie hier Ihren Text
ein. Fügen Sie hier Ihren Text ein. Fügen Sie hier Ihren Text
ein. Fügen Sie hier Ihren Text ein.

\section{Schwachstellen und Verbesserungsmöglicheiten}
%=====================================================

Fügen Sie hier Ihren Text ein. Fügen Sie hier Ihren Text ein. Fügen
Sie hier Ihren Text ein. Fügen Sie hier Ihren Text ein. Fügen Sie hier
Ihren Text ein. Fügen Sie hier Ihren Text ein. Fügen Sie hier Ihren
Text ein. Fügen Sie hier Ihren Text ein. Fügen Sie hier Ihren Text
ein. Fügen Sie hier Ihren Text ein. Fügen Sie hier Ihren Text
ein. Fügen Sie hier Ihren Text ein. Fügen Sie hier Ihren Text
ein. Fügen Sie hier Ihren Text ein. Fügen Sie hier Ihren Text
ein. Fügen Sie hier Ihren Text ein. Fügen Sie hier Ihren Text
ein. Fügen Sie hier Ihren Text ein.

%========================================================================

\newpage
\thispagestyle{empty}

\section*{Quellen}

{\parindent0pt%
  
[1] Möller, Ralf: Skript Rechnerarchitektur, AG Technische Informatik,
Technische Fakultät, Universität Bielefeld, Wintersemester 2021/22

[2] Einstein, Albert: Zur Elektrodynamik bewegter K{\"o}rper. Annalen
der Physik 322(10):891-921, 1905

[3] Wikipedia entry ``Radix sort'', konsultiert am 20. Februar 2022,
\url{https://en.wikipedia.org/wiki/Radix_sort}

}
%=========================================================================

\vspace*{3cm}

\noindent Dies bitte löschen:

\noindent textwidth in cm: \printinunitsof{cm}\prntlen{\textwidth}

\noindent textheight in cm: \printinunitsof{cm}\prntlen{\textheight}

\end{document}
